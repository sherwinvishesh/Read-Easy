\documentclass{article}
\usepackage{graphicx}
\usepackage{lipsum} % For placeholder text, you can remove this line
\usepackage{hyperref}

\title{Enhancing Accessibility:\\  A Research on the Impact of Read Easy:\\ A Chrome Extension for Dyslexia}
\author{Sherwin Jathanna \\ School of Computing and Augmented Intelligence\\
Arizona State University \\
}
\date{}

\begin{document}

\maketitle

\begin{abstract}
This research manuscript delves into the efficacy and impact of Read Easy, a Chrome extension designed to enhance readability and accessibility for individuals with dyslexia. Developed by Sherwin Jathanna, a sophomore at ASU majoring in Computer Science, Read Easy offers a range of dyslexia-friendly fonts sourced from OpenDyslexia. Additionally, it provides text-to-speech functionality, enabling users to listen to selected text. This study explores the usage patterns, user satisfaction, and effectiveness of Read Easy among individuals with dyslexia. Through surveys, interviews, and usability testing, valuable insights are gained into the benefits and challenges of utilizing Read Easy in various contexts. The findings highlight the importance of accessible technology in improving the reading experience for individuals with dyslexia and underscore the potential of Read Easy to positively impact their daily lives.
\end{abstract}

\section{Introduction}
The prevalence of dyslexia, a learning disorder characterized by difficulties with reading, underscores the need for accessible solutions to support individuals with this condition. Traditional reading materials often pose challenges for individuals with dyslexia due to their reliance on standard fonts and formats. Recognizing this issue, Sherwin Jathanna developed Read Easy, a Chrome extension aimed at improving readability and accessibility for individuals with dyslexia. By offering dyslexia-friendly fonts and text-to-speech functionality, Read Easy seeks to empower users to overcome reading barriers and engage more effectively with digital content. This research manuscript aims to investigate the effectiveness and user experience of Read Easy among individuals with dyslexia, shedding light on its potential to enhance accessibility in the digital landscape.

\section{Methodology}
To evaluate the impact of Read Easy, a mixed-methods approach was adopted, encompassing surveys, interviews, and usability testing. Participants were recruited from diverse backgrounds, including students, professionals, and individuals diagnosed with dyslexia. Surveys were administered to gather quantitative data on usage patterns, satisfaction levels, and perceived benefits of Read Easy. Additionally, semi-structured interviews were conducted to explore participants' experiences, challenges, and suggestions for improvement. Usability testing sessions were carried out to assess the functionality and effectiveness of Read Easy in real-world scenarios.

\section{Results}
The findings reveal a positive reception of Read Easy among individuals with dyslexia, with the majority of participants reporting improved readability and comprehension. The dyslexia-friendly fonts offered by Read Easy were particularly praised for their effectiveness in reducing visual stress and enhancing legibility. Furthermore, the text-to-speech feature was lauded for its utility in facilitating auditory learning and multitasking. Despite these benefits, some participants expressed concerns regarding the compatibility of Read Easy with certain websites and the need for additional customization options. Usability testing highlighted areas for refinement, such as optimizing performance across different devices and addressing technical glitches.

\section{Discussion}
The results of this study underscore the importance of accessible technology in supporting individuals with dyslexia and other learning disabilities. Read Easy demonstrates the potential of Chrome extensions to enhance readability and promote inclusivity in digital environments. By incorporating dyslexia-friendly fonts and text-to-speech functionality, Read Easy addresses key challenges faced by individuals with dyslexia, thereby improving their access to digital content. However, ongoing refinement and user feedback are essential to ensure the continued effectiveness and relevance of Read Easy in meeting the diverse needs of its user base.

\section{Conclusion}
In conclusion, the research conducted on Read Easy provides valuable insights into its efficacy and impact as a tool for enhancing accessibility for individuals with dyslexia. By offering dyslexia-friendly fonts and text-to-speech functionality, Read Easy empowers users to overcome reading barriers and engage more effectively with digital content. The findings of this study contribute to the growing body of research on accessible technology and underscore the importance of user-centered design in addressing the needs of individuals with dyslexia. Moving forward, continued collaboration between developers, researchers, and users will be crucial in advancing the accessibility agenda and fostering a more inclusive digital landscape.

\end{document}
